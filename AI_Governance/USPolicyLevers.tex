\documentclass{article}
\usepackage[utf8]{inputenc}
\usepackage{hyperref}
% Set margins
\usepackage[top=2cm, left=2cm, right=2cm, bottom=2cm]{geometry}
\usepackage{enumitem}
\setlist[itemize,2]{label=$\circ$}
% Use the multicol package to set the document to two columns
\usepackage{multicol}
\setlength{\columnsep}{20pt}

% Use a readable font
\usepackage{palatino}

% Set spacing
\setlength{\parindent}{0pt} % No paragraph indentation
\setlength{\parskip}{6pt} % Add space between paragraphs


\title{A Brief Summary of \textit{AI Policy Levers:
A Review of the U.S. Government’s Tools to Shape AI Research, Development, and Deployment} \\ \large Original by Sophie-Charlotte Fischer, Jade Leung, Markus Anderljung, Cullen O’Keefe, Stefan Torges, Saif M. Khan, Ben Garfinkel, and Allan Dafoe }

\author{Summary by Angus Robinson}
\date{21 March 2023}

\begin{document}

% Use the multicol environment to set the document to two columns

\maketitle
\begin{multicols}{2}

\section{Overview}

This report investigates how the U.S. government (USG) might influence AI research, development, and deployment within the U.S. and abroad, given its national security concerns. It provides an overview of the USG's policy levers within the current legal framework and assesses the plausibility of their future application to AI technologies. \\ Find the original (from March 2021) at \href{https://www.governance.ai/research-paper/ai-policy-levers-a-review-of-the-u-s-governments-tools-to-shape-ai-research-development-and-deployment}{https://www.governance.ai/research-paper/ai-policy-levers-a-review-of-the-u-s-governments-tools-to-shape-ai-research-development-and-deployment}.

\section{Key Definitions}

\begin{itemize}
\item \textbf{AI systems:} Machines capable of sophisticated information processing, including machine learning (ML) techniques and "good old fashioned AI" (GOFAI) paradigms.
\item \textbf{AI technologies:} AI systems and computer hardware enabling the production and use of AI systems.
\item \textbf{Secrecy orders:} A government order to conceal information in patent applications.
\item \textbf{Antitrust measures:} Measures designed to prevent monopolies.
\end{itemize}


\section{Policy Levers}

The following policy levers are covered in descending order of likelihood of use for explicit national security purposes:

\subsection{Measures already in use to some extent}
\begin{itemize}
\item \textbf{Federal R\&D funding:} USG could use funding to advance domestic AI development, secure access to intellectual property, or develop AI systems for national security applications.
\item \textbf{Foreign investment restrictions:} The Committee on Foreign Investment in the United States (CFIUS) reviews potential national security implications of foreign investments in U.S. companies and real estate transactions. On its recommendation, several Chinese bids to buy US chip manufacturers have already been blocked.
\item \textbf{Export controls:} USG currently applies export controls to a small number of AI technologies and the semiconductor supply chain. The Department of Commerce is reviewing whether to expand these controls.
\item \textbf{Visa vetting:} USG could use visa vetting procedures to deny entry to AI researchers, students, and workers from rival countries.
\end{itemize}
\subsection{Measures unlikely to be used extensively} 
\begin{itemize}
\item \textbf{Expanded visa pathways:} USG could expand or create visa programs to recruit foreign AI researchers and professionals, either for general purposes or specific national security projects. This might strengthen the domestic industry, but makes espionage more likely.
\item \textbf{Secrecy orders:} The Invention Secrecy Act authorises the U.S. Patent and Trademark Office to prevent the disclosure of information in patent applications when deemed detrimental to national security. The role of secrecy orders in AI governance is likely to be limited. It is possible these have had limited use already.
\end{itemize}
\subsection{Measures very unlikely to be used unless there is a vast increase in National Security Concerns}
\begin{itemize}
\item \textbf{Pre-publication screening procedures for security-sensitive publications:} USG could introduce voluntary or mandatory pre-publication screening procedures for AI research to prevent the spread of security-sensitive information, but implementation would be difficult given the AI research community's culture of openness.
\item \textbf{The Defense Production Act:} The Act confers upon the president authorities to require private companies to supply products, materials, and services for national defense. The USG could use this Act to prioritize AI-specific chip design and fabrication or recruit top AI researchers into the National Defense Executive Reserve.
\item \textbf{Born Secret Doctrine:} USG could announce that all information relevant to a particular technology is automatically classified, as with nuclear weapons. 
\end{itemize}
\subsection{May be used, but not for national security reasons}
\begin{itemize}
\item \textbf{Antitrust enforcement:} The USG could break up large AI companies for social or economic reasons, but it is unlikely they would do this for reasons of national security, as it would likely be illegal, and probably wouldn't help.
\end{itemize}

\section{Policy levers by Goal Pursued}
\begin{itemize}
\item \textbf{Strengthen the domestic AI industry:} federal R\&D funding; expanded visa pathways; careful antitrust enforcement.
\item \textbf{Weaken a rival's AI industry:} foreign investment restrictions; export controls; expanded visa pathways (to improve the US workforce; expanded visa vetting (to reduce opportunites for espionage).
\item \textbf{Make relevant AI technology available for national security purposes:} federal R\&D funding; targeted antitrust measures; Defense Production Act recruitment of AI researchers.
\item \textbf{Undermine a rival’s ability to use AI technology:} foreign investment restrictions; export controls; visa vetting; voluntary pre-publication screening; secrecy orders; a Born Secret doctrine.
\item \textbf{Gain insights into the AI R\&D landscape:} federal R\&D funding.
\item \textbf{Deprive a rival of insights into the AI R\&D landscape:} foreign investment restrictions; visa vetting; voluntary pre-publication screening; secrecy orders; a Born Secret doctrine

\end{itemize}

\section{Policy levers by potential downside}
\begin{itemize}
\item \textbf{Backlash from private actors:} pre-publication screening; Defense Production Act measures; secrecy orders; a Born Secret doctrine; antitrust measures.
\item \textbf{Increased espionage:} expanded visa pathways.
\item \textbf{Legal challenges:} secrecy orders; a Born Secret doctrine; antitrust measures.
\item \textbf{Weakening of domestic AI industry:} foreign investment restrictions, export controls, visa vetting, antitrust measures, a Born Secret doctrine; secrecy orders.
\item \textbf{Diplomatic backlash:} foreign investment restrictions; export controls.

\section{Conclusion}
\begin{itemize}
    \item The USG is already using certain levers, such as increasing federal R\&D funding for AI, tightening foreign investment restrictions, implementing export controls on AI technologies, and increasing visa vetting for foreign professionals and researchers.
    \item The use of secrecy orders in the context of AI is unclear but likely limited in scale.
    \item AI and semiconductor industries are calling for expanded visa pathways, but changes will depend on the current administration's stance on foreign worker inflow.
    \item Voluntary screening procedures for AI publications, the Defense Production Act, and antitrust action for national security purposes are less likely unless concrete and compelling threats to national security arise.
\end{itemize}

\end{itemize}
\end{multicols}
\end{document}
