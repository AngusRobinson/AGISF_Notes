\documentclass{article}
\usepackage[utf8]{inputenc}
\usepackage{hyperref}
% Set margins
\usepackage[top=2cm, left=2cm, right=2cm, bottom=2cm]{geometry}
\usepackage{enumitem}
\setlist[itemize,2]{label=$\circ$}
% Use the multicol package to set the document to two columns
\usepackage{multicol}
\setlength{\columnsep}{20pt}

% Use a readable font
\usepackage{palatino}

% Set spacing
\setlength{\parindent}{0pt} % No paragraph indentation
\setlength{\parskip}{6pt} % Add space between paragraphs


\title{A Summary of Nate Soares's \textit{Four Background Claims}}

\author{Summary by Angus Robinson}
\date{27 March 2023}

\begin{document}

% Use the multicol environment to set the document to two columns

\maketitle
\begin{multicols}{2}

\section{Overview}

Soares gives these four claims as a brief introduction to why AI Alignment is important: (1) General reasoning ability exists; (2) Artifical reasoners could be more intelligent than humans; (3) If they are, they will have an immense impact; (4) They will not be beneficial by default.
 \\ Find the original (from 2015) at: \\ \href{https://intelligence.org/2015/07/24/four-background-claims/}{https://intelligence.org/2015/07/24/four-background-claims/}.

\section{The Claims}
\subsection{Claim \#1: Humans have a very general ability to solve problems and achieve goals across diverse domains.}
\begin{itemize}
\item This is what we mean by General Intelligence.
\item Capabilities in domains alien to prehistory suggests generality rather than simply separate areas of narrow intelligence.
\item Evolution going from chimp-level intelligence to human-level intelligence relatively quickly suggests human engineers may not need many key insights to achieve this generality artificially.
\end{itemize}

\subsection{Claim \#2: AI systems could become much more intelligent than humans.}
\begin{itemize}
\item Brains are physical systems, and the input/output behaviour of any physical system is (probably) in principle computable.
\item Again, the short jump in evolution suggests the fundamental algorithms for general intelligence aren't impossible to find.
\item It would be an unlikely coincidence if humans are near some conceptual limit to intelligence, but in any case, merely running a human intelligence more quickly would effectively result in higher intelligence.
\item Intelligent machines could improve the world at an unprecedented pace.
\end{itemize}

\subsection{Claim \#3: If we create highly intelligent AI systems, their decisions will shape the future.}
\begin{itemize}
\item Humans have shaped the world with their intelligence; an even more intelligent system would be even more able to shape it.
\item A sufficiently technologically superior system has no need to coordinate or integrate with human society.
\begin{itemize}
\item In the worst case, it might simply use humans for the resources they provide.
\end{itemize}
\item It is prudent to prioritise research into syst
\end{itemize}

\subsection{Claim \#4: Highly intelligent AI systems won’t be beneficial by default.}
\begin{itemize}
\item Regardless of their intelligence, computers do what they are programmed to do.
\begin{itemize}
\item If you want compassion, you have to programme it in.
\begin{itemize}
\item e.g. a machine programmed to cure cancer might kidnap humans for experiments because it's the best way to cure cancer.
\end{itemize}
\end{itemize}
\item Programming in desirable intentions is a separate technical problem to improving the capabilities of the system.
\end{itemize}

\section{Conclusion}
\begin{itemize}
\item Four core claims in brief: (1) General reasoning ability exists; (2) Artifical reasoners could be more intelligent than humans; (3) If they are, they will have an immense impact; (4) They will not be beneficial by default.
\item Enormous financial and person resources are going into improving capabilities, relatively little into safety research.
\item Artificial superintelligences will probably arrive at some point, and depending on humanity, will have either an extremely positive or an extremely negative impact.
\end{itemize}

\end{multicols}
\end{document}
